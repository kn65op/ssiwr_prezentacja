\section{Wstęp}
\nextoc 

%---------------------------------------------------------------------------
\subsection{Mobilne zastosowania rozszerzonej rzeczywistości}
\begin{frame}
    \frametitle{Mobilna rozszerzona rzeczywistość}

    Są to wszelkie aplikacje działające na urządzeniach mobilnych (smartfony, tablety, itp.), które pozwalają na interakcję z użytkownikiem i otaczającym środowiskiem (np. obraz z kamery).

    Pozwala na poszerzenie wiedzy ogólnej, uzyskanie informacji o otoczeniu itp. bezpośrednio na ekranie urządzenia.
\end{frame}

\subsection{Zalety i wady}
\begin{frame}
    \frametitle{Zalety i wady}
    Zalety użycia urządzeń mobilnych w ramach poszerzonej rzeczywistości:
    \begin{itemize}
        \item Można mieć ,,cały świat w kieszenii''
        \item Szybkie uzyskanie interesujących informacji
    \end{itemize}
    Wady użycia urządzeń mobilnych w ramach poszerzonej rzeczywistości:
    \begin{itemize}
        \item Mała wydajność obliczeń (postęp technologiczny, działanie w chmurze)
        \item Konieczność transmisji danych i bycia w zasięgu sieci
    \end{itemize}
\end{frame}

